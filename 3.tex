\documentclass[11pt,a4paper]{article}
 
\usepackage{polski}
\usepackage[utf8]{inputenc} 
   % by użyć polskich znaków w systemach Linux
   % używamy kodowania "latin2" lub "utf8", dla Windows "cp1250" 
\title{LaTeX}    %tytuł tabeli
\author{}
\date{}
\begin{document}
  \maketitle
LSdlfjSLIDjfhDjhfDKJhfLDxjfjhgdzjcfhvDKJDFg
\begin{tabular}{|l|l|l|} %lub np. {lcrrr}, {|rr|} itd. w zależności od liczby żądanych kolumn i ich justowania
\hline %linia pozioma w tabeli
1111 & 1111... &1111 ... \\     %nagłówki tabeli
\hline %linia pozioma w tabeli
1111 & 1111... & 1111 ... \\	%treść tabeli
1111 & 1111... & 1111 ... \\
1111 & 1111... &1111 ... \\
1111 & 1111... &1111 ... \\
1111 & 1111... &1111 ... \\
1111 & 1111... &1111 ... \\
1111 & 1111... &1111 ... \\
1111 & 1111... &1111 ... \\
1111 & 1111... &1111 ... \\
1111 & 1111... &1111 ... \\
1111 & 1111... &1111 ... \\
1111 & 1111... &1111 ... \\
1111 & 1111... &1111 ... \\
1111 & 1111... &1111 ... \\
1111 & 1111... &1111 ... \\
1111 & 1111... &1111 ... \\
1111 & 1111... &1111 ... \\
1111 & 1111... &1111 ... \\
1111 & 1111... &1111 ... \\
1111 & 1111... &1111 ... \\
1111 & 1111... &1111 ... \\
1111 & 1111... &1111 ... \\
\hline %linia pozioma w tabeli
\end{tabular}
\end{document}
